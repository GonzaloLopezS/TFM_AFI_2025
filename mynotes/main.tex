\documentclass{article}
\usepackage{graphicx} % Required for inserting images

\title{Transporte marítimo e incautación de droga}
\author{Gonzalo Lopez Segovia}
\date{December 2024}

\begin{document}

\maketitle

\section{Introducción}
Historia del transporte marítimo. Transporte marítimo de mercancías. Logística.
Comercio mundial, nodos y puertos marítimos.

Tráfico de productos ilegales. Qué se considera como droga. La lucha contra el narcotráfico.
¿Cómo se mide la cantidad de mercancías? En containers (TEU), en tonelaje total transportado, en
cuantía del valor de la mercancía. ¿Cuál se adaptará mejor al problema a modelar?

\subsection{Motivación del proyecto}
Interés en la Geografía como elemento de estudio y su comprensión para generar valor con los datos estudiados a través de ella.


\subsection{Antecedentes}
Otros trabajos (¿mencionar TFM de Juande?), estudios pasados sobre geografía del transporte.
Las diferentes aproximaciones en la lucha contra el narcotráfico en lugares de destino. Enfoque particular para Estados Unidos de América (EE UU), Reino Unido (UK) y la Unión Europea.
SOTA in Drug Seizure and Maritim Traffic and Trade.

\section{Descripción del problema}
Diseño del Modelo

\section{Estado del arte}
Estudio con series temporales.
Al faltar datos de flujos y vías marítimas, no es trivial hallar la red o grafo mundial de conexiones portuarias.


\section{Implementación Práctica del Modelo}
Descripción de las tareas realizadas para implementar el proyecto.
Se destacan las principales actividades y decisiones.
Fases de desarrollo del proyecto.
Diagrama de Gantt con las tareas realizadas.
\subsection{Pre procesamiento de los datos}
Captura de los datos. Open Data vs Datos bajo Licencia. Organizaciones e instituciones. Datasets disponibles.
EDA, Limpieza, tratamiento de outliers.

Tratamiento de Variables (Feature Selection, Feature Engineering).
Visualizacion de datos.

¿Las series temporales de los datos? ¿Son o no son invariantes en el tiempo?¿Existe o no estacionalidad? ¿Su dependencia está ligada con el ciclo económico?
Conocimientos de trabajos pasados que ayuden a simplificar el número de observaciones y/o variables.
\subsection{Algoritmos de Machine Learning}
Agrupamiento de los barcos de EE UU mediante técnicas de clustering. Objetivo: hallar punto en el litoral que sirva como "punto de puerto" para los barcos más próximos. Es preferible que se trate de un algoritmo de clustering de tipo jerárquico: para ir aumentando la granularidad basándose en la situación geográfica.
Algoritmos tipo K-Means o PAM quedan descartados por dos motivos: no son jerárquicos y, sobre todo, su centro de masas tenderá a ser escogido en alta mar (o como punto medio o mediano de los barcos que formen cada clúster).

Para los puertos en Reino Unido no hay que realizar ninguna operación de este tipo. Sin embargo, a la hora de relacionar los distintos puertos y áreas portuarias con las regiones de autoridad policial, no es trivial. ¿Realizar embedding que relacione los puertos/ciudades con las regiones policiales?

\section{Metodología empleada}

\section{Resultados y Análisis}

\section{Conclusiones y trabajo futuro}

\section{Bibliografia}

\section{Glosario}

\end{document}
